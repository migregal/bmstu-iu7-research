\chapter{Общие сведения}

\section{Распознавание}

Стоит отдельно отметить, что в русском языке термин <<распознавание>> является достаточно объемным и включает в себя, непосредственно, обнаружение, классификацию и кластеризацию распознанных объектов.

В данной работе речь идет о рассматриваемой задаче распознавания объектов на аэрофотоснимке, т.\,е. изображении. Изображения можно получить в трех основных диапазонах: видимом, инфракрасном и радиоволновом спектрах. При этом, способ представления данных на изображении зависит от конкретной задачи -- это может быть трехканальный RGB, одноканальная градация серого и т.д.

Так же, существует несколько подходов к распознаванию объектов на изображении в зависимости от имеющихся данных.

Например, последовательность изображений содержит больше данных, чем единичное изображение. Благодаря этому можно повысить точность распознавания, применяя методы распознавания объектов в последовательности сигналов, учтя, тем самым, контекст происходящего на изображениях последовательности. При этом, последовательность может представлять собой как снимки одного и того же объекта с разных ракурсов, так и содержать различные состояния объекта в результате многократной съемки со стационарной позиции.

К тому же, при наличии карт глубины изображения или же изображения в альтернативном спектре можно комбинировать различные методы, повышая тем самым точность распознавания за счет использования всех источников данных~\cite{ensamble-methods}.

В любом случае, обнаружение объекта на изображении сводится к выделению области изображения, в которой находится объект. Обычно такие области задаются парами координат $(x_1,\,y_1)$ и $(x_2,\,y_2)$, задающие левый нижний и правый верхний углы области соответственно.

Результатом классификации, равно как и кластеризации, является разбиение распознанных объектов на группы. Такие группы принято называть классами. Основным отличием кластеризации является отсутствие необходимости в явном задании классов -- они формируются в процессе работы на основании признаков распознаваемых объектов, в то время как классификация подразумевает разбиение объектов на зараннее заданные классы.

В настоящее время основным методом распознавания каких-либо объектов является применение нейронных сетей.

В первую очередь, это связано с основным преимуществом нейронных сетей перед альтернативными методами -- отсутствие необходимости в явном задании признаков распознаваемых объектов.

\section{Марковские цепи с дискретным временем}

С помощью Марковских цепей можно представлять реальные процессы, являющиеся или приводимые к Марковским процессам. Процесс называется Марковским, если выполняется следующее равенство:

\begin{equation}
    \begin{split}
    P\{\xi(t_{n+1}) < x_{n+1}|\xi(t_{n}) &= \\ &= x_{n},\,\dots,\,\xi(t_{1}) = x_{1}\} = \\ &= P\{\xi(t_{n+1}) < x_{n+1}|\xi(t_{n}) = x_{n}\},
    \end{split}
\end{equation}

То есть условная функция распределения вероятностей значений $\xi(t_{n+1})$ для данного процесса в момент времени $t_{n+1}$ не зависит от значений процесса в
моменты времени $t_1,\,\dots,\,t_{n-1}$, а определяется лишь значением $\xi(t_{n})=x_n$ в момент времени $t_n$.

Для описания цепей Маркова можно использовать граф вероятностей
переходов. Вершины данного графа обозначают возможные состояния рассматриваемой системы, а стрелки от одной вершины к другой указывают возможные переходы
между состояниями. Число над стрелкой обозначает вероятность такого перехода. Например, пусть множество состояний $X=\{1,2,3\}$, а матрица вероятностей переходов имеет вид:

\begin{equation}
  P =
  \begin{bmatrix}
    1 & 0 & 0     \\
    1/2 & 0 & 1/2 \\
    2/3 & 0 & 1/3
  \end{bmatrix}.
\end{equation}

Тогда граф вероятностей переходов следующим образом:

\imgw{markov-chain}{h}{0.4\textwidth}{Пример графа вероятностей перехода}

Цепь Маркова полностью определяется матрицей вероятностей переходов и начальным распределением.

\section{Нейронные сети}

ANN определяется как массово параллельный распределенный процессор, состоящий из простых процессорных блоков, который обладает естественной склонностью накапливать эмпирические знания~\cite{ann}. 

Их название и структура вдохновлены человеческим мозгом, а алгоритм работы основывается на способе, которым биологические нейроны передают сигналы друг другу.

Простейшая ANN включает в себя входной слой, выходной (или целевой) слой и, между ними, скрытый слой. Слои соединены через узлы (искусственные нейроны). Эти соединения образуют <<сеть>> -- нейронную сеть -- из взаимосвязанных узлов. Пример приведен на рисунке~\ref{img:network}.

\imgw{network}{h}{0.6\textwidth}{Пример схемы ANN}

В общем случае искусственный нейрон можно образом, приведенном на рисунке~\ref{img:neuron}.

\imgw{neuron}{h}{0.6\textwidth}{Общая схема искусственного нейрона}

Данное представление применимо к любому виду нейронных сетей -- вне зависимости от типа, нейронные сети реализуются путем упорядочивания нейронов в слои и последующим связыванием соответствующих слоев между собой.~\cite{ann}

В процессе обучения нейронной сети используется так называемая обучающая выборка -- заранее подготовленный набор данных, отражающий суть рассматриваемой предметной области~\cite{ann}. В зависимости от содержимого обучающей выборки результирующие весовые конфигурации нейронной сети (т.~е. веса связей между нейронами, а так же смещения отдельно взятых нейронов) могут отличаться~\cite{ann}. В связи с этим одна и та же структура нейронной сети может переиспользована для работы в различных предметных областях.

\section{Сверточные нейронные сети}

Одним из основных видов неронных сетей, применяемых для распознавания является CNN~\cite{cnn}.

CNN представляет собой тип ANN, которая имеет архитектуру с глубокой обратной связью и выделяется на фоне остальных ANN с полносвязными слоями своей способностью к обобщению. CNN работает с сильно абстрагированными характеристиками объектов, особенно это касается пространственных данных, что позволяет добиться более эффективно идентифицировать объекты в сравнении с другими типами ANN~\cite{cnn}. Одним из отличительных свойств CNN является способность к фильтрации посторонних шумов во входных данных.

Модель CNN состоит из конечного набора уровней обработки, которые могут изучать различные характеристики входных данных (например, изображения) с несколькими уровнями абстракции. Начальные уровни изучают и извлекают высокоуровневые свойства, а более глубокие уровни изучают и извлекают более низкоуровневые свойства. Базовая концептуальная модель CNN была показана на рисунке~\ref{img:cnn-conceptual-model}.

\imgw{cnn-conceptual-model}{h}{0.6\textwidth}{Концептуальная модель CNN}

% , различные типы слоев описаны в последующих разделах.

Существующие архитектуры CNN для обнаружения объектов на изображениях можно разделить на две категории: одноэтапные (one-stage) и двухэтапные (two-stage)~\cite{review-on-one-stage-object-detection}.    

\subsection{Свертка}

\subsubsection*{Ядро}

Прежде, чем рассматривать процесс свертки, необходимо определить понятие <<ядро>>, используемое при свертке. Ядро предстваляет собой сетка из дискретных значений или чисел, где каждое значение известно как вес этого ядра. Пример двухмерного ядра приведен на рисунке~\ref{img:convolution-kernel}.

\imgw{convolution-kernel}{h}{0.1\textwidth}{Пример двумерного ядра с размерностью $2 \times 2$}

Ядро инициализируется случайными значениями, которые изменяются в ходе обучения CNN.

\subsubsection*{Пример свертки}

Разберем пример свертки для изображения в градациях серого, т.к. такое изображение содержит лишь один канал, передаваемый на вход CNN.

Пусть дано изображение в градациях серого, представленное на рисунке~\ref{img:convolution-input-example}.

\imgw{convolution-input-example}{h}{0.2\textwidth}{Пример изображения в градациях серого с размерностью $4 \times 4$}

Далее рассмотрим первые два шага процесса свертки, представленные на рисурках~\ref{img:convolution-example-1}~и~\ref{img:convolution-example-2}, соответсвенно.

\imgw{convolution-example-1}{h}{0.6\textwidth}{Пример изображения в градациях серого с размерностью $4 \times 4$}

\imgw{convolution-example-2}{h}{0.6\textwidth}{Пример изображения в градациях серого с размерностью $4 \times 4$}

Аналогичным образом свертка продолжается до полного заполнения результирующей матрицы. Стоит отметить, что, в зависимости от размера окна и размера перекрытия окон, будет меняться размер результирующей матрицы.

\subsection{Двухэтапные алгоритмы}

В таких нейросетевых алгоритмах выделяют два этапа: поиска RoI (англ. Candidate Region Extraction) на изобаржении и последующей классификации RoI, найденных на первом этапе. При этом под RoI на изображении подразумеваются зоны, потенциально содержащие искомые объекты~\cite{overview-of-two-stage-object-detection}. 

\imgw{two-step-cnn}{h}{0.6\textwidth}{Схема работы двухэтапного алгоритма}

%\includeimage{two-step-cnn}{f}{h}{0.6\textwidth}{Схема работы двухэтапного алгоритма}

Стоит отметить, что первый этап может происходить без использования нейронных сетей. Для этого можно использовать информациию о контрасте, ключевые точки или перебор всех возможных положений объекта с помощью процедуры~\texttt{selective search}~\cite{realtime-recognition-algorythm}.

RoI, полученные вышеперечисленными методами, могут обладать серьезными недостатками, например:
\begin{itemize}
    \item[---] содержать слишком большое количество фона;
    \item[---] содержать лишь небольшую часть объекта;
    \item[---] содержать более одного объекта.
\end{itemize}

В связи с этим на первом этапе более предпочтительным методом является применение CNN, не содержащих полносвязных слоев~\cite{realtime-recognition-algorythm}.

На втором этапе CNN применяеются к обнаруженным RoI.

Преимуществом данных алгоритмов является высокая точность распознавания объектов, однако, платой за это является время, необходисмое для выделения <<подозрительных>> зон на изображении~\cite{overview-of-two-stage-object-detection}.

\subsection{Одноэтапные алгоритмы}

Данные нейросетевые алгоритмы не включают в себя этап поиска RoI на изображении.

\imgw{one-step-cnn}{h}{0.6\textwidth}{Схема работы одноэтапного алгоритма}

%\includeimage{one-step-cnn}{f}{h}{0.6\textwidth}{Схема работы одноэтапного алгоритма}

Преимуществами одноэтапных алгоритмов являются их простота и относительно высокая скорость работы. К недостаткам же можно отнести более низкую точность детектирования объектов по сравнению с двухэтапными алгоритмами, а также меньшую гибкость алгоритма с точки зрения рассматриваемых изображений~\cite{review-on-one-stage-object-detection}.

